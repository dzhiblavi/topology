\section{База топологии}

\begin{definition}
	\textit{Базой топологии} $\O$ называют $\Sigma \subseteq \O$ такое, что
\[
	\forall U \in \O~ \exists \l_\a \in \Sigma\colon~ U = \bigcup{\l_\a}
\]
\end{definition}

\begin{theorem}
	$\Sigma$ --- база топологиии $\O$ тогда и только тогда, когда
\[
	\forall x \in X~ \forall U(x)~ \exists V \in \Sigma \colon x \in V \subseteq U
\]
\end{theorem}
\begin{proof}
	\enewline
	\begin{itemize}
		\item[$\Lra$] $\Sigma$ --- база топологии, поэтому
\[
	\exists \l_\a \in \Sigma\colon~ U(x) = \bigcup{\l_\a}
\]
		Поэтому $\exists \a\colon x \in \l_\a$.
		\item[$\Lla$] Пусть $A$ открыто, тогда
\[
	A = \bigcup_{x \in A}{V(x)}
\]
	\end{itemize}
\end{proof}

\begin{definition}
	$\Sigma_x$ называется \textit{базой топологии в точке $x$}, если
	\begin{itemize}
		\item $\forall V \in \Sigma_x~ x \in V$.
		\item $\forall U \in \O\colon~ x \in U~ \exists V \in \Sigma_x\colon x \in V 
			\subseteq U$.
	\end{itemize}
\end{definition}

\begin{remark}
	$\Sigma$ --- база топологии тогда и только тогда, когда
\[
	\forall x \in X~ \Sigma_x = \{\, U \in \Sigma \mid x \in U \,\}
	\text{ --- база топологии в } x
\]
\end{remark}

\begin{remark}
	$\forall x~ \Sigma_x$ --- базы в точках, тогда
	$\bigcup{\Sigma_x}$ --- база топологии.
\end{remark}

\begin{theorem}
	$\Sigma$ --- база некоторой топологии тогда и только тогда, когда
	\begin{itemize}
		\item $X = \bigcup{\Sigma}$.
		\item $\forall U, V \in \Sigma~ U \cap V$ представляется в виде
			объединения элементов $\Sigma$.
	\end{itemize}
\end{theorem}
\begin{proof}
	\enewline
	\begin{itemize}
		\item[$\Lra$] $X$ открыто, поэтому обязательно выполнено первое
			условие. Второе условие выполнено потому,
			что $U \cap V$ открыто, то есть по определению должно представляться
			в виде объединения элементов $\Sigma$.
		\item[$\Lla$] Назначим $\O$ как всевозможные объединения множеств из $\Sigma$.
			Проверим, что $\O$ --- топология.
			\begin{itemize}
				\item $\varnothing, X \in \O$ --- очевидно.
				\item $\bigcup{U} \in X$ по построению.
				\item $U \cap V = \left(\bigcup{U_i}\right) \cap 
					\left(\bigcup{V_i}\right) = \bigcup{U_i \cap V_i}$.
					$U_i \cap V_i$ открыто по посылке, поэтому $U \cap V$
					открыто.
			\end{itemize}
	\end{itemize}
\end{proof}
