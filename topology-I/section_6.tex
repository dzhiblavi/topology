\section{Связность}

\begin{definition}
	\textit{Связным} называется пространство, которое нельзя разбить на два
	непустых непересекающихся открытых (замкнутых) множества.
\end{definition}

\begin{remark}
	Связность эквивалентна условию: не существует непрерывного сюрьективного отображения
	$f \colon X \to \{\,0, 1\,\}$.
\end{remark}
\begin{proof}
	\enewline
	\begin{itemize}
		\item[$\Lra$] Если бы такое отображение существовало,
			$X$ можно было бы разбить на открытые множества $f^{-1}(\{\,1\,\})$
			и $f^{-1}(\{\,0\,\})$. Они непусты, так как $f$ сюрьективно.
		\item[$\Lla$] Предположим, что пространстно можно разбить на два
			открытых непустых нерересекающихся множества $A$, $B$. Тогда положим
\[
	f(x) = \begin{cases}
		0,~ x \in A \\
		1,~ x \in B
	\end{cases}
\]
			Это отображение непрерывно и сюрьективно, потому что множества $A$ и $B$
			открыты и непусты.
	\end{itemize}
\end{proof}

\begin{theorem}
	Непрерывный образ связного пространства связен.
\end{theorem}
\begin{proof}
	Предположим, что это не так, то есть $f \colon X \to Y$, $X$ связно и $f$ непрерывно,
	но $f(X)$ несвязно. В таком случае, существует непрерывное сюрьективное отображение
	$g \colon f(X) \to \{\,0, 1\,\}$. Но тогда отображение $g \circ f$ непрерывно
	и сюрьективно, то есть $X$ несвязно:
\[
	\xymatrix{
		X \ar[r]^{f} \ar@/_1pc/@{-->}[rr]_{g \circ f} & f(X) \ar[r]^{g} & \{\,0, 1\,\}
	}
\]
\end{proof}

\begin{theorem}
	Отрезок $[0, 1]$ связен.
\end{theorem}
\begin{proof}
	Предположим обратное и разобъем отрезок на два непересекающихся замкнутых
	множества $A$ и $B$. Предположим не ограничиваю общности, что $0 \in A$.
	Тогда пусть $b = \inf{B}$. Множества замкнуты, поэтому $b \in B$ и $b \neq 0$.
	В таком случае $[0, b) \subseteq A$, из чего следует, что $b \in A$.
\end{proof}

\begin{definition}
	Множество $Y \subseteq X$ называется \textit{связным}, если оно связно как
	индуцированное топологическое пространство.
\end{definition}

\begin{theorem}(Характеризация связных множеств на прямой)

	На прямой связны только интервалы.
\end{theorem}
\begin{proof}
	\enewline
	\begin{itemize}
		\item[$\Lra$] Предположим, что интервал несвязен. Тогда
			возьмем соответствующее разбиение $A$, $B$ и пересечем его с
			$[a, b]$, где $a$, $b$ взяты из интервала:
\begin{align*}
	A' = A \cap [a, b] \\
	B' = B \cap [a, b]
\end{align*}
			Тем самым мы получили, что $[a, b]$ несвязен, чего быть не может.
		\item[$\Lla$] Предположим, что $Y$ --- не интервал. Тогда по определению
			интервала сущестуют $a$, $b$, $c$ такие, что $(a, b) \subseteq Y$, $c \notin Y$.
			Тогда положим $A = Y \cap (-\infty, c)$, $B = Y \cap (c, +\infty)$ --- разбиение
			$Y$ (оба множества не пусты из-за наличия в них $a$ и $b$ соответственно). 
			Поэтому $Y$ несвязно.
	\end{itemize}
\end{proof}

\begin{theorem}(О среднем значении)
	Пусть $X$ --- связное топологическое пространство, $f \colon X \to \R$ непрерывно.
	Тогда $\forall a, b \in f(X)~ [a, b] \subseteq f(X)$.
\end{theorem}
\begin{proof}
	$f(X)$ связно в $\R$, поэтому является интервалом, откуда незамедлительно
	следует утверждение теоремы.
\end{proof}

\begin{lemma}
	Замыкание связного множества связно.
\end{lemma}
\begin{proof}
	Предположим обратное и разобъем замыкание на два непересекающихся непустых
	открытых в замыкании множества: $\Cl(A) = U \cup V$. Положим
\begin{align*}
	U' = A \cap U \\
	V' = A \cap V
\end{align*}
	Эти множества открыты в $A$ по определению индуцированной из $\Cl(A)$ топологии.
	Осталось только проверить, что эти множества непусты. Пусть $p \in U \subseteq A$,
	поэтому $p$ --- точка прикосновения $A$, что дает $\forall U(p)~ U(p) \cap A \neq 
	\varnothing$, откуда $U'$ непусто. Аналогично получаем, что $V'$ непусто.
\end{proof}

\begin{theorem}
	Объединение попарно пересекающихся связных множеств связно.
\end{theorem}
\begin{proof}
	Пусть $A = \bigcup{A_i}$ несвязно, тогда разобъем его: $A = U \cup V$.
	Пусть теперь $p \in U$, $p \in A_i$, $q \in V$, $q \in A_j$. Поскольку
	$A_i$ связно, $A_i \cap V = \varnothing$ (иначе бы получилось разбиение
	$A_i = (A_i \cap U) \cup (A_i \cap V)$). Аналогично $A_j \cap U = \varnothing$,
	откуда $A_i \cap A_j = \varnothing$, что противоречит условию.
\end{proof}

\begin{definition}
	\textit{Компонентой связности} точки $p$ называется объединение всех
	связных множеств, содержащих $p$.
\end{definition}

\begin{lemma}(Свойства компонент связности)
	\enewline
	\begin{itemize}
		\item Компоненты связности замкнуты.
		\item Компоненты связности не пересекаются, то есть $X$ разбивается на
			компоненты связности.
		\item $p \sim q \Llra \exists$ связное $A \colon p \in A$, $q \in A$
			--- отношение эквивалентности.
	\end{itemize}
\end{lemma}
\begin{proof}
	\enewline
	\begin{itemize}
		\item Множество всегда можно замкнуть увеличив его (если оно не замкнуто) и
			сохранив при этом связность.
		\item Если бы компоненты пересекались, их можно было бы объединить в одну,
			сохранив связность и увеличив при этом множество.
		\item Следует из предыдущих утверждений.
	\end{itemize}
\end{proof}

\section{Компактность}

\begin{definition}
	Топологические пространство называется \textit{компактным}, если из любого его
	открытого покрытия можно выбрать конечное подпокрытие.
\end{definition}

\begin{theorem}
	Отрезок в $\R$ компактен.
\end{theorem}
\begin{proof}
	Предположим противное: пусть $U_i$ --- покрытие отрезка открытыми множествами,
	из которого нельзя выбрать конечное. Разделим отрезок пополам. Для одной из
	половин нельзя будет выбрать конечное покрытие. Заменим отрезок на эту половину
	и продолжим этот процесс (отрезок на $k$-м шаге обозначим $I_k$).
	Из аксиомы Кантора найдется точка $p \in \bigcap{I_k}$. Пусть $p \in U_{i_0}$,
	тогда при больших $k$ получаем, что $I_k \subseteq U_{i_0}$, что дает конечное
	покрытие для $I_k$.
\end{proof}

\begin{theorem}
	Любое замкнутое подмножество компакта --- компакт.
\end{theorem}
\begin{proof}
	Пусть $A \subseteq X$ замкнуто, $U_i$ --- открытое покрытие $A$.
	Тогда $U_i \cup \overline{A}$ --- открытое покрытие $X$. Выберем из него
	конечное подпокрытие ($X$ компакт). Выкинем из этого подпокрытия $\overline{A}$,
	если оно туда попало. Получим конечное подпокрытие для $A$.
\end{proof}

\begin{remark}
    Конечное объединение компактов --- компакт.
\end{remark}

\begin{theorem}
	Прямое произведение компактов --- компакт.
\end{theorem}
\begin{proof}
	Для доказательства достаточно рассмотреть покрытие $X \times Y$ базовыми
	окрестностями $U \times V$. $Y \simeq \{\,x_0\,\} \times Y$ --- компакт,
	поэтому можно выбрать конечное подпокрытие этого множества:
\[
	\{\,x_0\,\} \times Y \subseteq \bigcup_{k = 1}^{n}{U_k \times V_k}
\]
	Рассмотрим теперь покрытие $X$ множествами $W_{x_0} = \bigcap_{k = 1}^{n}{U_k}$,
	где $U_k$ взяты из покрытия соответствующего $\{\,x_0\,\} \times Y$. Выберем
	из получившегося покрытия конечное:
\[
	X \subseteq \bigcup_{k = 1}^{m}{W_{x_k}}
\]
	Получается, что $\forall x \in X~ W_x \times Y$ покрывается конечным набором
	множеств покрытия, поэтому
\[
	X \times Y \subseteq \bigcup_{k = 1}^{m}{W_{x_k} \times Y}
\]
	Есть конечное покрытие $X \times Y$.
\end{proof}

\begin{theorem}(Характеризация компактов в $\Rm$)

	В $\Rm$ компактность равносильна замкнутости и ограниченности.
\end{theorem}
\begin{proof}
	\enewline
	\begin{itemize}
		\item[$\Lra$] Накроем компакт всевозможными шарами, вытащим оттуда
			конечное покрытие, получим ограниченность. Замнкутость доказывается
			в следующей теореме.
		\item[$\Lla$] По последней теореме имеем, что $[a, b]$ компактен в $\Rm$.
			Любое ограниченное множество можно вписать в подобный параллелепипед,
			поэтому это множество --- замкнутое подмножество компакта, то есть компакт.
	\end{itemize}
\end{proof}

\begin{theorem}
	В хаусдорфовом пространстве компакты замкнуты.
\end{theorem}
\begin{proof}
	Докажем открытость дополнения компакта $K$. Пусть $x \notin K$. Покажем,
	что $x$ входит в $K^c$ с некоторой окрестностью. Для каждой точки
	$y \in K$ из хаусдорфовости построим окрестности
\[
	\exists U(y), U(x)\colon~ U(y) \cap U(x) = \varnothing
\]
	$U(y)$ образуют конечное покрытие $K$. Выберем из него конечное:
\[
	K \subseteq \bigcup_{k = 1}^{n}{U(y_k)}
\]
	Тогда $\bigcap{U_k(x)}$ открыто как конечное пересечение открытых, причем
	по построению $\bigcap{U_k(x)} \cap K = \varnothing$, поэтому $x$ лежит
	в $K^c$ с этой окрестностью.
\end{proof}

\begin{theorem}
	Любое хаусдорфово компактное пространство регулярно.
\end{theorem}

\begin{theorem}
	Любое хаусдорфово компактное пространство нормально.
\end{theorem}
\begin{proof}
	Последние две теоремы доказываются по очереди так же, как и теорема
	о замкнутости, только в каждой следующей теореме надо пользоваться предыдущей.
\end{proof}

\begin{theorem}(Вейерштрасс)

	Пусть $X$ --- хаусдорфово топологическое пространство, $f \colon X \to \R$
	непрерывно, тогда $\exists \max{f}, \min{f}$.
\end{theorem}
\begin{proof}
	Теперь очевидно.
\end{proof}

\begin{theorem}
	Пусть $X$ --- компакт, $Y$ --- хаусдорфово т.п., $f \colon X \to Y$ ---
	непрерывная биекция, тогда $f$ гомеоморфизм.
\end{theorem}
\begin{proof}
	Все, что надо доказать, чтобы $f$ стало гомеоморфизмом, это то, что $f^{-1}$
	непрерывно. Проверим, что под действием $f^{-1}$ прообразы замкнутых множеств
	замкнуты. Пусть $A \subseteq X$ --- замкнуто, то есть и компактно. Прообраз
	этого множества под действием $f^{-1}$ в точности совпадает с $f(A)$ ---
	непрерывным образом компакта, то есть компактом, то есть замкнутым множеством,
	что и требовалось. 
\end{proof}

\begin{corollary}
	Непрерывное инъективное отображение компакта в хаусдорфово пространство
	всегда является топологическим вложением.
\end{corollary}

\section{Линейная связность}

\begin{definition}
	\textit{Путем} в топологическом пространстве $X$ называется непрерывное
	отображение отрезка в $X$, то есть $C([a, b]) \ni \gamma \colon [a, b] \to X$.
\end{definition}

\begin{definition}
	Топологическое пространство называется \textit{линейно связным},
	если любые две его точки можно соединить путем.
\end{definition}

\begin{lemma}
	Пусть для любых двух точек $x, y \in X$ существует связное множество
	$I \subseteq X$, содержащее $x, y$. Тогда $X$ связно.
\end{lemma}
\begin{proof}
	Предположим обратное и разобъем $X$ на открытые $A$, $B$. Выберем
	точки $a$, $b$ из $A$ и $B$ соотрветственно. Пусть $I$ --- то связное множество,
	существование которого для точек $a$ и $b$ утверждается в условии.
	Тогда рассмотрим разбиение $I = (I \cap A) \cup (I \cap B)$. Множества
	разбиения непусты, непересекаются и открыты в топологии, индуцированной в
	$I$ из $X$, из чего можно сделать вывод, что $I$ несвязно.
\end{proof}

\begin{theorem}
	Всякое линейно связное пространство связно.
\end{theorem}
\begin{proof}
	Для доказательства просто предъявим для произвольных двух точек $X$
	связное множество, содержащее эти две точки. Пусть $x, y \in X$,
	соединим их путем $\gamma \colon [a, b] \to X$. Отрезок связен,
	путь как отображение непрерывен, поэтому носитель пути тоже связен,
	то есть подходит в качестве связного множества, содержащего $x, y$.
\end{proof}

\begin{theorem}
	Непрерывный образ линейно связного пространства линейно связен.
\end{theorem}
\begin{proof}
	Пусть $f \colon X \to f(X) \subseteq Y$ --- непрерывное отображение
	линейно связного пространства $X$. Докажем, что $f(X)$ линейно связно.
	Пусть $x, y \in f(X)$, причем $x = f(a)$, $y = f(b)$ для каких-либо
	$a, b \in X$. Соединим путем $\gamma \colon [0, 1] \to X$, $\gamma(0) = a$
	, $\gamma(1) = b$. Тогда подходящим путем в $f(X)$ будет отображение 
	$f \circ \gamma \colon [0, 1] \to f(X)$.
\end{proof}

\begin{remark}
	В общем случае из связности \textbf{не следует} линейная связность.
\end{remark}

\begin{remark}
	Отношение $\sim$ на $X$, в котором две точки эквивалентны, если
	они могут быть соединены путем, является отношением эквивалентности.
\end{remark}
\begin{proof}
	Очевидно.
\end{proof}

\begin{definition}
	Классы эквивалентности по только что введенному отношению называются
	\textit{компонентами линейной связности}.
\end{definition}

