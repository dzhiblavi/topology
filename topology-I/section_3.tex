\section{Индуцированная топология}

\begin{definition}
	Пусть $X$ --- топологическое пространство, $Y \subset X$. Тогда
	на $Y$ можно завести топологию, которую называют \textit{индуцированной}:
	множество $A \subset Y$ открыто тогда и только тогда, когда
	$\exists B \in \O(X)\colon~ A = B \cap Y$. В таком случае
	топологическое пространство $\langle Y, \O(Y) \rangle$ 
	называют \textit{подпространством}.
\end{definition}

\begin{theorem}(База подпространства в точке)

	Пусть $X$ --- топологическое пространство, $Y \subset X$ --- его подпространство,
	$y \in Y$, $\Sigma_y$ --- база $X$ в точке $y$. Тогда
\[
	\{\, U \cap Y \,\}_{U \in \Sigma_y}
\]
	является базой индуцированной топологии в точке $y$.
\end{theorem}
\begin{proof}
	Пусть $y \in A$ открыто в $Y$. Тогда по определению индуцированной топологии
	$\exists U \in \O(X)\colon~ A = U \cap Y \ni y$. Тогда по определению
	базы в точке $\exists V \in \Sigma_y\colon~ y \in V \subseteq U$.
	Тогда $y \in V \cap Y \subseteq U \cap Y$, что и требовалось.
\end{proof}

\begin{corollary}
	Пусть $\Sigma$ --- база $\O(X)$, $Y$ --- подпространство $X$.
	Тогда 
\[ 
	\{\, V \cap Y \mid V \in \Sigma \,\} \text{ --- база } \O(Y)
\]
\end{corollary}

\begin{theorem}
	Пусть $X$ --- т.п., $Y \subset X$ --- его подпространство, $A \subseteq Y$.
	Тогда
	\begin{itemize}
		\item $open_X(A) \Lra open_Y(A)$.
		\item $cl_Y(A) \Llra \exists cl_X(B)\colon~ B \cap Y = A$.
		\item $cl_X(A) \Lra cl_Y(A)$.
		\item $open_X(Y) \Llra (open_Y(A) \Llra open_X(A))$.
		\item $cl_X(Y) \Llra (cl_Y(A) \Llra cl_X(A))$.
	\end{itemize}
\end{theorem}

\begin{theorem}(О транзитивности индуцирования)

	Пусть $X$ --- т.п., $X \supset Y \supset Z$, тогда
	топологии $Y \to Z$, $X \to Z$ совпадают.
\end{theorem}
\begin{proof}
\begin{align*}
	open_{X \to Z}(A) \Llra \exists open_X(U)\colon~ A = U \cap Z \Llra
	A = \underbrace{(U \cap Y)}_{open_Y} \cap Z \Llra open_{Y \to Z}(A).
\end{align*}	
\end{proof}

\begin{remark}
	\enewline
	\begin{itemize}
		\item $\Cl_X(A) \cap Y = \Cl_Y(A)$.
		\item $\Int_X(A) \cap Y \neq \Int_Y(A)$ (вообще говоря).
	\end{itemize}
\end{remark}
