\chapter{Основные понятия}

\section{Метрическое пространство}

\begin{definition}
    \textit{Метрикой} на множестве $X$ называют $\rho \colon X \to \mathbb{R}$,
    удовлетворяющую аксиомам метрики:
    \begin{itemize}
        \item $\rho(x) \geqslant 0$
        \item $\rho(x, y) = \rho(y, x)$
        \item $\rho(x, y) + \rho(y, z) \geqslant \rho(x, z)$
    \end{itemize}
\end{definition}

\begin{definition}
    Пару $\langle X, \rho \rangle$, где $\rho$ --- метрика на $X$, называют
    \textit{метрическим пространством}
\end{definition}

\begin{examples}
    \enewline
    \begin{itemize}
        \item Стандартная метрика на $\mathbb{R}^n$: $\rho(x, y) = |x, y|_2$,
        где $d_k(x, y) \defeq |x, y|_k = \sqrt[k]{\sum_{i=1}^{n}(x_i - y_i)^k}$
        \item $|., .|_k$ является метрикой на $\mathbb{R}$ при любых $k
        \geqslant 1$
        \item $|x, y|_{\infty} = \max_{i=1}^{n}(x_i - y_i)$ --- метрика
        на $\mathbb{R}$
        \item $\rho(x, y) = 1$ при $x \neq y$ и $\rho(x, y) = 0$ иначе ---
        метрика, порождающая дискретное пространство.
    \end{itemize}
\end{examples}

\textit{Далее, если не указано, речь идет о метрическом пространстве $X$}

\begin{definition}
    \textit{Шаром} радиуса $r$ с центром в точке $x$ называется
\[
    B_r(x) \defeq \{\, y \in X \mid \rho(x, y) < r \,\}
\]
\end{definition}

\begin{definition}
    \textit{Замкнутым шаром} радиуса $r$ с центром в точке $x$ называется
\[
    \overline{B_r}(x) \defeq \{\, y \in X \mid \rho(x, y) \leqslant r \,\}
\]
\end{definition}

\begin{definition}
    \textit{Расстоянием} от точки $x$ до множества $A$ называется
\[
    \rho(x, A) \defeq \inf_{y \in A}{\rho(x, y)}
\]
\end{definition}

\begin{definition}
    \textit{Диаметром} множества $A$ называется
\[
    \diam(A) = \sup{\{\, \rho(x, y) \mid x, y \in A \,\}}
\]
\end{definition}

\begin{definition}
    В метрическом пространстве \textit{открытыми} называют множества $A$
    такие, что
\[
    \forall x \in A~ \exists B_r(x) \subset A
\]
Иначе говоря, любая точка открытого множества входит в него с некоторым шаром.
\end{definition}

\begin{definition}
    Множество $A$ называют \textit{ограниченным}, если $\diam(A) < +\infty$
\end{definition}

\begin{theorem}
    Множество $A$ ограниченно $\Llra$ его можно вписать в шар
\end{theorem}
\begin{proof}
    \enewline
    \begin{itemize}
        \item[$\Lra$] Пусть $m = \diam(A)$. Покажем, что $A$ можно вписать в
        шар радиуса $m + 1$. Возьмем произвольную точку $x \in A$. Тогда
        $\forall y \in A~\rho(x, y) \leqslant m < m + 1 \Lra y \in B_{m+1}(x)$
        \item[$\Lla$] Пусть $y, z \in A$ и $A$ можно вписать в шар $B_r(x)$.
        Тогда $2r > \rho(x, y) + \rho(x, z) \geqslant \rho(y, z) \Lra \rho(y, z)
        < 2r \Lra A$ ограничено.
    \end{itemize}
\end{proof}

\begin{theorem}
    \enewline
    \begin{itemize}
        \item Произольное объединение открытых множеств открыто
        \item Пересечение двух (а значит, и произвольного конечного числа)
        открытых множеств открыто.
    \end{itemize}
\end{theorem}
\begin{proof}
    \enewline
    \begin{itemize}
        \item Пусть $\{\, G_\a \,\}_{\a \in A}$ --- семейство открытых множеств. Тогда
        \begin{gather*}
            x \in \bigcup_{\a \in A}{G_\a} \Lra x \in G_\a \Lra \exists U(x)
            \subset G_\a \subset \bigcup_{\a \in A}{G_\a}
        \end{gather*}
        \item Пусть $A$ и $B$ --- открытые множества. Тогда
        \begin{gather*}
            x \in A \cap B \Lra x \in A ~\wedge~ x \in B \Lra \\
            \exists B_{r_1}(x) \subset A ~\wedge~ B_{r_2}(x) \subset B \Lra \\
            x \in B_{\min(r_1, r_2)}(x) \subset A \cap B
        \end{gather*}
    \end{itemize}
\end{proof}

\begin{definition}
    \textit{Липшицево эквивалентными} называют отображения $f$ и $g$ в
    $\mathbb{R}$, такие, что $\exists c_1, c_2 \colon~ c_1f \leqslant g \leqslant
    c_2f$
\end{definition}
\begin{example}
    В $\mathbb{R}^n$ метрики $d_1$ и $d_2$ липшицево эквивалентны
\end{example}

\newpage

\section{Топологическое пространство}

\begin{definition}
    \textit{Топологией} на множестве $X$ называют $\O \subseteq \mathcal{P}(X)$,
    удовлетворяющее следующим свойствам:
    \begin{itemize}
        \item $\varnothing, X \in \O$
        \item $A, B \in \O \Lra A \cap B \in \O$
        \item $\displaystyle \{\, X_\a \in \O \,\}_{\a \in A} \Lra
        \bigcup_{\a \in A}{X_a}\in \O$
    \end{itemize}
    Иными словами, топология замкнута относительно конечных пересечений и
    произвольных объединений её элементов.
\end{definition}

\begin{definition}
    Пара $\langle X, \O \rangle$, где $\O$ --- топология на $X$, называется
    \textit{топологическим пространством}.
\end{definition}

\begin{definition}
    Элементы топологии называются \textit{открытыми множествами}. Дополнения
    открытых множеств называются \textit{замкнутыми множествами}.
\end{definition}

\begin{examples}
    \enewline
    \begin{itemize}
        \item $\O = \mathcal{P}(X)$ --- дискретная топология
        \item $\O = \{\, \varnothing, X \,\}$ --- антидискретная
        топология
        \item Все метрические пространства являются топологическими
        пространствами, порожденными метрикой.
        \item $\O = \varnothing \cup \{\, \text{все дополнения конечных
        множеств} \,\}$
    \end{itemize}
\end{examples}

\begin{definition}
    \textit{Метризуемым} называется топологическое пространство, топология
    которого может быть порождена метрикой.
\end{definition}

\begin{examples}
    \enewline
    \begin{itemize}
        \item Дискретная топология метризуема
        \item Антидискретная топология не метризуема
    \end{itemize}
\end{examples}

\begin{definition}
    \textit{Окрестностью} точки $x$ называют любое открытое множество, содержащее
    $x$. Далее окрестность точки $x$ будет обозначаться $U(x)$.
\end{definition}

\begin{definition}
    Точка $x$ называется \textit{внутренней} для множества $A$, если она входит в
    него с некоторой окрестностью:
\[
    \exists U(x) \colon~ U(x) \subset A
\]
\end{definition}

\begin{definition}
    Точка $x$ называется \textit{граничной} точкой множества $A$, если любая окрестность
    точки $x$ имеет непустое пересечение как с $A$, так и с его дополнением:
\[
    \forall U(x)~~ A \cap U(x) \neq \varnothing \wedge (X \setminus A) \cap U(x)
    \neq \varnothing
\]
\end{definition}

\begin{definition}
    Точка $x$ называется \textit{внешней} точкой $A$, если
\[
    \exists U(x)~~ A \cap U(x) = \varnothing
\]
\end{definition}

\begin{definition}
	Точка $x$ называется \textit{точкой прикосновения (предельной точкой)}
	множества $A$, если
\end{definition}
\[
    \forall U(x)~~ A \cap U(x) \neq \varnothing
\]

\begin{remark}
    Точка прикосновения и внешняя точка --- формальные отрицания друг друга.
\end{remark}

\begin{theorem}
    \enewline
    \begin{itemize}
        \item $\varnothing$, $X$ замкнуты
        \item $A$, $B$ замкнуты $\Lra$ $A \cup B$ замкнуто
        \item если $C_\a$ замнкнуты, то $\bigcap_{\a \in A}{C_\a}$ замкнуто
    \end{itemize}
\end{theorem}
\begin{proof}
    \enewline
    \begin{itemize}
        \item $X = X \setminus \varnothing$ --- замкнуто по опделелению.
        Аналогично $\varnothing = X \setminus X$
        \item $A \cup B$ замкнуто $\Llra X \setminus (A \cap B)$ открыто
        $\Llra (X \setminus A) \cup (X \setminus B)$ открыто $\Lla (X \setminus
        A)$, $(X \setminus B)$ открыты $\Llra A$, $B$ замкнуты.
        \item Аналогично ii
    \end{itemize}
\end{proof}

\begin{theorem}
    $A$ открыто, $B$ замкнуто. Тогда

    \begin{itemize}
        \item $A \setminus B$ открыто
        \item $B \setminus A$ замкнуто
    \end{itemize}
\end{theorem}
\begin{proof}
    \enewline
    \begin{itemize}
        \item $A \setminus B = A \cap (X \setminus B)$ --- открыто
        \item $B \setminus A = B \cap (X \setminus A)$ --- замкнуто
    \end{itemize}
\end{proof}

\section{Внутренность и замыкание}

\begin{definition}
    \textit{Внутренностью} множества $A$ называют наибольшее по включению
    открытое множество, содержащееся в $A$, иначе говоря:
\[
    \Int(A) \defeq \bigcup_{\substack{U \subseteq A \\open_X(U)}}{U}
\]
\end{definition}

\begin{definition}
    \textit{Замыканием} множества $A$ называют наименьшее по включению
    замкнутое множество, сожержащее $A$, иначе говоря:
\[
    \Cl(A) \defeq \bigcap_{\substack{C \supseteq A \\cl_X(C)}}{C}
\]
\end{definition}

\begin{theorem}(Свойства $\Int$)
    \enewline
    \begin{itemize}
        \item $\Int(A)$ открыто
        \item $\Int(A) \subseteq A$
        \item $open_X(B),\, B \subseteq A \Lra B \subseteq \Int(A)$
        \item $\Int(A) = A \Llra open_X(A)$
        \item $\Int(\Int(A)) = A$
        \item $A \subseteq B \Lra \Int(A) \subseteq \Int(B)$
        \item $\Int(A \cap B) = \Int(A) \cap \Int(B)$
        \item $\Int(A \cup B) \supseteq \Int(A) \cup \Int(B)$
    \end{itemize}
\end{theorem}
\begin{proof}
    \enewline
    \begin{itemize}
        \item $\Int(A)$ открыто как объединение открытых
        \item В объединения входят только подмножества $A$, поэтому
        $\Int(A) \subseteq A$
        \item $B$ по определению войдет в объединение
        \item $\Lra$ по пункту (i). $\Lla$ по пункту (iii)
        \item см. пункт (iv)
        \item Все открытые подмножества $A$ являются открытыми
        подмножествами $B$
        \item $A \cap B \subseteq A,~ A \cap B \subseteq B \Lra \\
        \Int(A \cap B) \subseteq \Int(A), \Int(B) \Lra \Int(A \cap B) \subseteq
        \Int(A) \cap \Int(B)$ \\ \\
        $\Int(A) \cap \Int(B) \subseteq \Int(A) \subseteq A$, аналогично
        $\Int(A) \cap \Int(B) \subseteq B$, поэтому \\ $\Int(A) \cap \Int(B)
        \subseteq A \cap B \Lra \Int(\Int(A) \cap \Int(B)) = \Int(A \cap B) \Lra
        \Int(A) \cap \Int(B) \subseteq \Int(A \cap B)$
    \end{itemize}
\end{proof}

\begin{theorem}(Свойства $\Cl$)
    \enewline
    \begin{itemize}
        \item $\Cl(A)$ замкнуто
        \item $\Cl(A) \supseteq A$
        \item $cl_X(B),\, B \supseteq A \Lra B \supseteq \Cl(A)$
        \item $\Cl(A) = A \Llra cl_X(A)$
        \item $\Cl(\Cl(A)) = A$
        \item $A \subseteq B \Lra \Cl(A) \subseteq \Cl(B)$
        \item $\Cl(A \cup B) = \Cl(A) \cup \Cl(B)$
        \item $\Cl(A \cap B) \subseteq \Cl(A) \cap \Cl(B)$
    \end{itemize}
\end{theorem}
\begin{proof}
    Можно доказать аналогично предыдущей теореме, а можно доказать, пользуясь
    переходом к дополнению в предыдущей теореме.
\end{proof}

\begin{theorem}(Связь $\Int$ и $\Cl$)
    \begin{itemize}
        \item $X \setminus \Int(A) = \Cl(X \setminus A)$
        \item $X \setminus \Cl(A) = \Int(X \setminus A)$
    \end{itemize}
\end{theorem}
\begin{proof}
    \enewline
    \begin{itemize}
        \item $$ X \setminus \Int(A) \defeq X \setminus
        \big( \bigcup_{\substack{U \subseteq A \\open_X(U)}}{U} \big) =
        \bigcap_{\substack{U \subseteq A \\open_X(U)}}{X \setminus U} \defeq
        \Cl(X \setminus A) $$
        так как множества вида $X \setminus U$ суть замкнутые множества,
        содержащие $A$
        \item Аналогично
    \end{itemize}
\end{proof}

\begin{definition}
    \textit{Границей} множества $A$ называется
\[
    \Fr(A) \defeq \Cl(A) \setminus \Int(A)
\]
\end{definition}

\begin{theorem}(Свойства $\Fr$)
    \enewline
    \begin{itemize}
        \item $\Fr(A)$ замкнуто
        \item $\Fr(A) = \Fr(X \setminus A)$
        \item $A$ замкнуто $\Llra \Fr(A) \subseteq A$
        \item $A$ открыто $\Llra \Fr(A) \cap A = \varnothing$
    \end{itemize}
\end{theorem}
\begin{proof}
    \enewline
    \begin{itemize}
        \item Очевидно в свете предыдущих теорем
        \item $A$ замкнуто $\Llra \Cl(A) = A \Llra \Cl(A) \setminus \Int(A)
        \subseteq A$
        \item $A$ открыто $\Llra \Int(A) = A \Llra \Fr(A) = \Cl(A) \setminus
        A \Llra \Fr(A) \cap A = \varnothing$
    \end{itemize}
\end{proof}

\begin{theorem}(Характеризация внутренности)

    $\displaystyle \Int(A)$ --- множество всех внутренних точек $A$.
\end{theorem}
\begin{proof}
    Докажем, что $x \in \Int(A) \Llra x$ --- внутренняя точка $A$

    \item[$\Lra$] $x \in \Int(A)$ --- открыто $\Lra U(x) = \Int(A) \subseteq A
    \Lra x$ --- внутренняя точка $A$
    \item[$\Lla$] $x$ --- внутренняя для $A \Lra \exists U(x) \subseteq A
    \Lra x \in \Int(A)$ так как по определению $\Int(A)$ --- это объединение всех
    открытых множеств, содержащихся в $A$, в том числе и $U(x)$.
\end{proof}

\begin{corollary}

    $A$ открыто $\Llra \forall x \in A~ x$ --- внутренняя точка $A$
\end{corollary}

\begin{theorem}(Характеризация замыкания)

    $\displaystyle \Cl(A)$ --- множество всех точек прикосновения $A$.
\end{theorem}
\begin{proof}
\[
    X \setminus \Cl(A) = \Int(X \setminus A) = \{\, \text{внешние точки } A \,\} =
    X \setminus \{\, \text{точки прикосновения } A \,\}
\]
\end{proof}

\begin{definition}
    Множество $A$ называется \textit{всюду плотным}, если $\Cl(A) = X$.
\end{definition}

\begin{definition}
    Топологическое пространство $X$ называют \textit{сепарабельным}, если
    в нем существует не более чем счетное всюду плотное множество.
\end{definition}

\begin{remark}
	Всюду плотность множества $A$ эквивалентна
	\begin{itemize}
		\item $\Int(X \setminus A) = \varnothing$.
		\item $\forall open_X(D)~ D \cap A \neq \varnothing$.
	\end{itemize}
\end{remark}
\begin{proof}
	\enewline
	\begin{itemize}
		\item $\Int(X \setminus A) = X \setminus \Cl(A) = \varnothing$.
		\item Если это условие не выполнилось для какого-то непустого $D$,
			то любая его точка является внешней для множества $A$, а значит,
			не входит в замыкание. Если же это условие выполнилось для всех $D$, 
			то любая окрестность любой точки пересекается с $A$ (надо
			взять $D = $ этой окрестности), значит, любая точка является 
			точкой прикосновения $A$, о есть $A$ всюду плотно.
	\end{itemize}
\end{proof}

\begin{definition}
	Множество $A \subseteq X$ называют \textit{нигде не плотным}, если
	внутренность его замыкание пуста: $\Int(\Cl(A)) = \varnothing$.
\end{definition}

\begin{remark}
	Нигде не плотность множества $A$ эквивалентна тому, что
	в любом непустом открытом множестве найдется открытое подмножество,
	не пересекающееся с $A$.
\end{remark}

\begin{theorem}
	В сепарабельном пространстве не существует более чем счетного
	дизъюнктного набора непустых открытых множеств.
\end{theorem}
\begin{proof}
	Пусть $U_i$ --- более чем счетный дизъюнктный набор непустых открытых
	множеств. Выберем тогда из каждого $U_i$ точку $p_i$, которая
	лежит в пересечение $U_i \cap S$, где $S$ --- какое-нибудь
	счетное всюду плотное множество. Получим, что $\{\, p_i \,\} \subseteq S$,
	то есть $S$ более, чем счетно.
\end{proof}

\section{Сравнение топологий}

\begin{definition}
	Пусть $\O_1$, $\O_2$ --- топологии на $X$. Говорят, что топология 
	$\O_1$ \textit{слабее} топологии $\O_2$, если $\O_1 \subset \O_2$.
\end{definition}

\begin{theorem}(Сравнение метрических топологий)

	Пусть $d_1$, $d_2$ --- метрики на $X$, $Top(d)$ --- топология,
	порожденная метрикой $d$. Тогда $Top(d_1) \subseteq Top(d_2)$
	тогда и только тогда, когда в любом шаре по $d_1$ содержится 
	шар по $d_2$ c таким же центром.
\end{theorem}
\begin{proof}
	\enewline
	\begin{itemize}
		\item[$\Lra$] Раз $Top(d_1) \subseteq Top(d_2)$, то шар $B_1(x, r)$ по метрике 
			$d_1$ открыт в $Top(d_2)$, значит любая его точка, включая $x$, 
			входит в $B_1(x, r)$ с некоторой окрестностью $B_2(x, r')$ во второй 
			топологии.
		\item[$\Lla$] Проверим, что открытое в первой топологии множество $U$ открыто
			во второй топологии. Для этого проверим, что все его точки 
			--- внутренние по второй метрике. $U$ открыто в $Top(d_1) \Lra
			\forall x \in U~ \exists~ B_1(x, r) \subseteq U 
			\Lra \exists~ B_2(x, r) \subseteq U$. 
	\end{itemize}
\end{proof}

\begin{corollary}
	$d_1 \leqslant d_2 \Lra Top(d_1) \subseteq Top(d_2)$.
\end{corollary}

\begin{corollary}
	$\exists c > 0\colon~ d_1 \leqslant c d_2 \Lra Top(d_1) \subseteq Top(d_2)$.
\end{corollary}

\begin{corollary}
	$d_1$, $d_2$ липшицево эквивалентны $\Lra Top(d_1) = Top(d_2)$.
\end{corollary}

