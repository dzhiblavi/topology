\section{Непрерывность}

\begin{definition}
	Отображение топологических пространств $f \colon X \to Y$ называется
	\textit{непрерывным}, если прообраз любого открытого множества открыт.
\end{definition}

\begin{remark}
	Открытость в определении можно заменить на замкнутость.
\end{remark}

\begin{remark}
	$f \colon X \to Y$ непрерыно $\Llra \forall B~ f^{-1}(\Int(B)) \subseteq 
	\Int(f^{-1}(B))$.
\end{remark}
\begin{proof}
	\enewline
	\begin{itemize}
		\item[$\Lra$] $f^{-1}(\Int(B)) = \Int(f^{-1}(\Int(B))) \subseteq \Int(f^{-1}(B))$
		\item[$\Lla$] Пусть $U$ открыто. Тогда 
			$f^{-1}(U) = f^{-1}(\Int(U)) \subseteq \Int(f^{-1}(U)) \subseteq f^{-1}(U)$.
			Поэтому $f^{-1}(U) = \Int(f^{-1}(U))$, что и означает, что $f^{-1}(U)$
			открыто.
	\end{itemize}
\end{proof}

%\begin{remark}
%	\enewline
%	\begin{itemize}
%		\item $f$ непрерывно $\Llra \forall A~ f(\Cl(A)) \subset \Cl(f(A))$.
%		\item $f$ непрерывно $\Llra \forall A~ f(\Int(A)) \supset \Int(f(A))$.
%	\end{itemize}
%\end{remark}

\begin{definition}
	Отображение называется \textit{открытым}, если образ открытого множества
	всегда открыт.
\end{definition}

\begin{remark}
	$f \colon X \to Y$ открыто $\Llra \forall A~ f(\Int(A)) \subseteq \Int(f(A))$.
\end{remark}
\begin{proof}
	Аналогично подобному замечанию о непрерывных отображениях.
\end{proof}

\begin{definition}
	Пусть $X \subset Y$, $X$ --- подпространство $Y$,
	тогда отображение 
\[
	in_{X \to Y} \colon X \to Y, ~in(x) = x
\]
	называется \textit{вложением} $X$ в $Y$.
\end{definition}

\begin{remark}
	Вложение непрерывно.
\end{remark}

\begin{theorem}(Композиция непрерывных отображений)

	Пусть $f \colon X \to Y$, $g \colon Y \to Z$ непрерывны. Тогда
	$g \circ f \colon X \to Z$ непрерывно.
\end{theorem}
\begin{proof}
	Пусть $U$ открыто в $Z$. Тогда
\[
	(g \circ f)^{-1}(U) = f^{-1}(g^{-1}(U))
\]
	Открыто.
\end{proof}

\begin{remark}
	Усиление и ослабление топологии оставляют непрерывные отображения непрерывными.
\end{remark}

\begin{theorem}(Непрерывность сужения)
	
	Если $f \colon X \to Y$ непрерывно, $Z \subseteq X$ --- подпространство $X$,
	тогда $f\big|_Z$ непрерывно.
\end{theorem}
\begin{proof}
	$f\big|_Z = f \circ in_{Z \to X}$ непрерывно как композиция непрерывных отображений.
\end{proof}

\begin{theorem}

	$f \colon X \to Y$ непрерывно $\Llra f(X) \subseteq Z \subseteq Y$,
	тогда $\hat{f} \colon X \to Z,~ \hat{f}(x) = f(x)$ непрерывно.
\end{theorem}
\begin{proof}
	\enewline
	\begin{itemize}
		\item[$\Lla$] $f$ непрерывно как композиция: $f =
			in_{Z \to Y} \circ \hat{f}$.
		\item[$\Lra$] Пусть $V$ открыто в $Z$. Тогда $V = U \cap Z$, где $U$
			открыто в $Y$. Поэтому
\[
	\hat{f}^{-1}(V) = f^{-1}(V) = f^{-1}(U \cap Z) = f^{-1}(U) \cap f^{-1}(Z)
	= f^{-1}(U) \text{ --- открыто в } X
\]
	\end{itemize}
\end{proof}

\begin{definition}
	$f \colon X \to Y$ непрерывно в $x \in X$, если
\[
	\forall U(f(x))~ \exists V(x)\colon~ f(V) \subset U
\]
\end{definition}

\begin{theorem}
	$f \colon X \to Y$ непрерывно $\Llra$ $\forall x \in X~ f$ непрерывно в $x$. 
\end{theorem}
\begin{proof}
	\enewline
	\begin{itemize}
		\item[$\Lra$] Для любой точки и ее окрестности $U$ можно взять $V = f^{-1}(U)$.
		\item[$\Lla$] Пусть $U$ открыто в $Y$, покажем, что $f^{-1}(U)$ открыто в $X$,
			то есть любая точка $f^{-1}(U)$ внутренняя.
\[
	x \in f^{-1}(U) \Lra \exists V(x)\colon~ f(V) \subset U \Llra V \subset f^{-1}(U)
	\Lra x \text{ --- внутренняя точка } f^{-1}(U)
\]
	\end{itemize}
\end{proof}

\begin{theorem}(Непрерывность в точке в терминах баз)
	
	Пусть $f \colon X \to Y$, $\Sigma_x$ --- база топологии в $x$,
	$\Lambda_{f(x)}$ --- база топологии	в $f(x)$. Тогда
\[
	f \text{ непрерывна в } x \Llra \forall U \in \Lambda_{f(x)}~
	\exists V \in \Sigma_x \colon f(V) \subset U
\]
\end{theorem}
\begin{proof}
	\enewline
	\begin{itemize}
		\item[$\Lra$] Пусть $U \in \Lambda_{f(x)}$, тогда $f^{-1}(U)$
			открыт, то есть в нем есть элемент базы $\Sigma_x$, что нам
			и нужно.
		\item[$\Lla$] Пусть $U$ открыто в $Y$, тогда в $U$ есть базовая окрестность 
			$\Lambda_i \ni f(x)$. Для этого элемента по посылке существует базовая
			окрестность $\Sigma_i \ni x \colon~ f(\Sigma_i) \subseteq \Lambda_i$.
			Эта $\Sigma_i$ и подходит под определение непрерывности в точке.
	\end{itemize}
\end{proof}

\begin{definition}
	\textit{Липшицевым} называется отображение метрических пространств
	$f \colon X \to Y$ такое, что $\exists C \colon~ \forall x_1, x_2 \in X~
	\r_Y(f(x_1), f(x_2)) \leqslant C \cdot \r_X(x_1, x_2)$.
\end{definition}

\begin{theorem}
	Все липшицевы отображения непрерывны.
\end{theorem}
\begin{proof}
	Зафиксируем базы топологий, состоящие из всевозможных шаров. Тогда
\[
	\forall \e > 0~ \exists \delta = \frac{\e}{C}\colon~ f(B_\delta(x)) \subseteq
	B_\e(f(x))
\]
	Что по теореме о непрерывности в терминах баз дает непрерывность в любой точке,
	а значит, и непрерывность на $X$.
\end{proof}

\begin{theorem}
	Пусть $f\colon\langle X, \O \rangle \to \langle Y, \Sigma\rangle$, $x \in X$.
	Топологическому пространству $\langle X, \O \rangle$ и точке $x$
	сопоставим топологическое пространство $\langle X, \O^x \rangle$:
\[
	\O^x = \{\, \varnothing, A \mid \exists U(x) \subseteq A  \,\}
\]
	Аналогично поступим и для $Y$ и $f(x)$. Тогда $f$ можно понимать как отображение между
	этими топологическими пространствами $\hat{f}$. В таком случае, верно утверждение:
\[
	f \text{ непрерывно в } x \Llra \hat{f} \text{ непрерывно}
.\]
\end{theorem}
\begin{proof}
	\enewline
	\begin{itemize}
		\item[$\Lra$] Пусть множество $B \subseteq Y$ открыто в $\Sigma^{f(x)}$. В таком
			случае, $\exists V(f(x)) \subseteq B$ открытое в $\Sigma$. Прообраз
			$V$ открыт по предположению, что $f$ непрерывно в точке, поэтому, раз
\[
	f^{-1}(V) \subseteq f^{-1}(B)
\]
			то множество $f^{-1}(B)$ открыто в $\O^x$, что и требовалось.
		\item[$\Lla$] Пусть теперь множество $f(x) \in B \subseteq Y$ открыто в $\Sigma$. Тогда
			оно автоматически открыто и в $\Sigma^{f(x)}$. Раз так, по посылке
			прообраз $B$ открыт в $\O^x$, что по определению дает существование
			требуемой для непрерывности в точки окрестности $x$.
	\end{itemize}
\end{proof}

\begin{corollary}
	Пусть $f$ непрерывно в $x$, $g$ непрерывно в $f(x)$. Тогда $g \circ f$ непрерывно в $x$.
\end{corollary}

\section{Гомеоморфизм}

\begin{definition}
	\textit{Гомеоморфизмом} топологических пространств $X$ и $Y$ называют
	биективное в обе стороны непрерывное отображение этих пространств. Другими
	словами, отображение сопоставляет открытым множествам открытые (биективно).
	$X$ и $Y$ в таком случае называют \textit{гомеоморфными}.
\end{definition}

\begin{definition}
	$f \colon X \to Y$ называется \textit{вложением} $X$ в $Y$, если оно
	осуществляет гомеоморфизм между $X$ и $f(X)$.
\end{definition}

\begin{definition}
	Покрытие $\Gamma$ пространства $X$ называется \textit{фундаментальным},
	если выполнено:
\[
	\forall A \subseteq X~~ [\forall C \in \Gamma~ open_C(A \cap C) \Lra open_X(A)]
\]
\end{definition}

\begin{remark}
	В определении, как всегда, можно везде заменить открытость на замкнутость.
\end{remark}

\begin{theorem}
	Пусть $\Gamma$ --- фундаментальное покрытие $X$, $f \colon X \to Y$, 
	$\forall C \in \Gamma~ f\big|_C$ непрерывно, тогда $f$ непрерывно на $X$.
\end{theorem}
\begin{proof}
	Пусть $U$ открыто в $Y$. Проверим, что $f^{-1}(U)$ открыто в $X$:
\[
	\forall C \in \Gamma~ f^{-1}(U) \cap C = \left(f\big|_C\right)^{-1}(U) 
\]
	Последнее множество открыто в $C$ потому, что $f\big|_C$ непрерывно.
	Тогда по самому определению фундаментального покрытия имеем, что
	$f^{-1}(U)$ открыто.
\end{proof}

\begin{theorem}
	\enewline
	\begin{itemize}
		\item Все открытые покрытия фундаментальны.
		\item Все конечные замкнутые покрытия фундаментальны.
		\item Все локально конечные замкнутые покрытия фундаментальны.
	\end{itemize}
\end{theorem}
\begin{proof}
	\enewline
	\begin{itemize}
		\item $A \cap C$ открыто в $C$ для всех $C \in \Gamma$, $C$ открыто в $X$,
			поэтому $A \cap C$ открыто в $X$. Раз так, $A = \bigcup{A \cap C}$
			открыто в $X$.
		\item Совершенно аналогично первому пункту (за исключением того, что
			пользуемся определением фундаментальности в терминах замкнутых множеств
			и того, что объединение только конечного числа замкнутых множеств замкнуто).
		\item За каждой точкой зафиксируем окрестность $U_x$, имеющую
			пересечение с конечным множеством множеств из $\Gamma$. $U_x$ ---
			фундаментальное покрытие $X$ (так как открытое). Пусть теперь
			$\forall C \in \Gamma~ A \cap C$ открыто в $C$. Тогда
			$\forall C \in \Gamma~ \forall x \in X~ A \cap C \cap U_x$ открыто в $C \cap U_x$
			по определению индуцированной в $C \cap U_x$ топологии. Тогда
			$(A \cap C) \cap (C \cap U_x)$ открыто в $C \cap U_x$, где
			$C \cap U_x$ образует конечное замкнутое покрытие $U_x$ (по построению). Это
			покрытие фундаментально по второму пункту, поэтому по определению его 
			фундаментальности, $A \cap U_x$ открыто в $U_x$. Применяя снова
			определение фундаментальности, только уже к покрытию $U_x$, получаем открытость
			$A$.
	\end{itemize}
\end{proof}

\section{Прямое произведение топологических пространств}

\begin{definition}
	Пусть $\langle X, \O \rangle$, $\langle Y, \Sigma \rangle$ --- топологические
	пространства. Тогда их прямым произведением называется топологическое пространство
	с носителем $X \times Y$, база топологии которого состоит из всевозможных множеств
	вида
\[
	A \times B,~ A \in \O, B \in \Sigma
\]
\end{definition}

\begin{lemma}
	Только что определенная система множеств действительно является базой топологии.
	Для этого достаточно проверить (теорема 1.5.2), что пересечение элементов базы
	представляется в виде объединения элементов той же базы:
\[
	(A \times B) \cap (A' \times B') = (A \cap A') \times (B \cap B')	
\]
	Здесь объединение состоит из одного элемента.
\end{lemma}

\begin{theorem}
	Прямое произведение замкнутых $A$, $B$ замкнуто в прямом произведении топологий.
	\textit{Замечание: прямое произведение открытых множеств открыто просто по определению}.
\end{theorem}
\begin{proof}
	Покажем, что $(X \times Y) \setminus (A \times B)$ открыто:
\[
	(X \times Y) \setminus (A \times B) = (X \setminus A) \times Y \cap X \times (Y \setminus B)
\]
\end{proof}

\begin{remark}
	Пусть $A \subseteq X$, $B \subseteq Y$. Тогда
	\begin{itemize}
		\item $\Cl(A \times B) = \Cl(A) \times \Cl(B)$.
		\item $\Int(A \times B) = \Int(A) \times \Int(B)$.
	\end{itemize}
\end{remark}
\begin{proof}
	Докажем второе утверждение:
\[
	\Int(A \times B) = \bigcup_{\substack{open_{X\times Y}(C) \\ C \subseteq A \times B}}{C}
	= \bigcup_{\substack{open_{X \times Y}(A' \times B') \\ A'\times B' \subseteq A \times B}}
	{A' \times B'} = \bigcup_{\substack{A' \subseteq A \\ open_X(A')}}{A'} 
	\times \bigcup_{\substack{B' \subseteq B \\ open_Y(B')}}{B'} = \Int(A) \times \Int(B)
\]
	Первый переход можно сделать, потому что любое открытое множество $C$ представляется
	в виде объединения множеств из базы.
\end{proof}

\begin{definition}
	Отображение
\[
	pr_x \colon X \times Y \to X,~ (x, y) \mapsto x
\]
	называется \textit{проекцией} $X \times Y$ на $X$.
\end{definition}

\begin{theorem}
	Проекция непрерывна.	
\end{theorem}
\begin{proof}
	Пусть $U$ открыто в $X$. Тогда $pr_X^{-1}(U) = U \times Y$ --- открыто в $X \times Y$
	по определению.
\end{proof}

\begin{theorem}
	Пусть $X$, $Y$, $Z$ --- топологические пространства, $f \colon Z \to X \times Y$,
	$f = (g, h)$, где $g \colon Z \to X$, $h \colon Z \to Y$. Тогда
\[
	f \text{ непрерывно } \Llra g, h \text{ непрерывны}
\]
\end{theorem}
\begin{proof}
	\enewline
	\begin{itemize}
		\item[$\Lra$] $g = pr_X \circ f$, $h = pr_Y \circ f$.
		\item[$\Lla$] Достаточно проверить открытость прообраза на базовых множествах:
\[
	f^{-1}(U \times V) = g^{-1}(U) \cap h^{-1}(V)
\]
	\end{itemize}
\end{proof}

\begin{corollary}
	\textit{Координатный слой} гомеоморфен $X$:
\[
	X \times \{\,y_0\,\} \simeq X
\]
\end{corollary}
\begin{proof}
	Установим гомеоморфизм
\[
	f \colon X \to X \times \{\,y_0\,\},~ x \mapsto (x, y_0)
\]
	Тогда
\[
	f^{-1} = pr_X\big|_{X \times \{\,y_0\,\}}
\]
	И оба предъявленных отображения непрерывны.
\end{proof}

\begin{theorem}
	Пусть $X$ --- т.п., $f, g\colon X \to \R$ непрерывны. Тогда
	$f + g$, $fg$, $\frac{f}{g}$ (при $g \neq 0$) непрерывны.
\end{theorem}
\begin{proof}
	Будем пользоваться тем, что арифметические операции непрерывны:
\begin{displaymath}
	\xymatrix{
		X \ar[r]^{(f, g)} & \R^2 \ar[r]^{+} & \R
	}
\end{displaymath}
	Здесь отображение $x \mapsto (f(x), g(x))$ непрерывно по последней теореме.
\end{proof}
