\section{Аксиомы отделимости}

\begin{definition}
	Топологическое пространство называется \textit{хаусдорфовым}, 
	если 
\[
		\forall x \neq y~ \exists~ U(x), U(y)\colon~ U(x) \cap U(y) = \varnothing
\]
	Хаусдорфовость --- вторая аксиома отделимости (Т2).
\end{definition}

\begin{remark}
	Все метрические пространства являются хаусдорфовыми.
\end{remark}

\begin{definition}
	Топологическое пространство удовлетворяет первой аксиоме
	отделимости (Т1), если 
\[
	\forall x \neq y~\exists~ U(x)\colon~ U(x) \not\ni y
\]
\end{definition}

\begin{theorem}
	Т.п. $X$ удовлетворяет Т1 тогда и только тогда, когда в нем
	любое одноточечное множество замкнуто.
\end{theorem}

\begin{definition}
	Множества $A$, $B$ называются отделимыми, если 
\[
	\exists U(A), U(B)\colon~ U(A) \cap U(B) = \varnothing
\]
\end{definition}

\begin{definition}
	Топологическое пространство называется \textit{регулярным} (Т3),
	если в нем выполняются свойства:
	\begin{itemize}
		\item Все одноточечные множества замкнуты (Т1).
		\item $\forall x~ \forall cl_X(A)\colon~ x \notin A~~ x$ отделима от $A$. 
	\end{itemize}
\end{definition}

\begin{remark}
	Регулярность эквивалентна набору свойств
	\begin{itemize}
		\item Все одноточечные множества замкнуты (Т1).
		\item $\forall x~ \forall U(x)~ \exists V(x)\colon~ \Cl(V) \subset U$
			($U$, $V$ открыты).
	\end{itemize}
\end{remark}
\begin{proof}
	\enewline
	\begin{itemize}
		\item[$\Lra$] Пусть $x \in A$ открыто, поэтому $\overline{A}$ замнкнуто, причем
			$x \notin \overline{A}$. Тогда $x$ отделима от $\overline{A}$:
\[
	\exists U(\overline{A}), U(x)\colon~ U(\overline{A}) \cap U(x) = \varnothing
\]
	Тогда можно взять $V = U(x)$: $\Cl(V) \cap \overline{A} = \varnothing$: если 
	бы $\Cl(V) \cap \overline{A} \ni p$, то 
\[
	\forall U(p)~ U(p) \cap U(x) \neq \varnothing
\]
	Но $p$ --- внутренняя точка $U(\overline{A})$, поэтому входит в нее с
	некоторой $U(p)$, которая пересекается с $U(x)$, чего быть не может.
		\item[$\Lla$] Снова перейдем к дополнению: множество $\overline{A}$
			открыто, причем $x \in \overline{A}$. Тогда
\[
	\exists V(x)\colon~ \Cl(V) \subset \overline{A}
\]
	Тогда можно отделить $A$ и $x$ множествами $X \setminus \Cl(V(x))$ 
	и $V(x)$ соответственно.
	\end{itemize}
\end{proof}

\begin{definition}
	Топологическое пространство называется \textit{нормальным} (Т4), если
	в нем любые два непересекающихся замкнутых множества отделимы.
\end{definition}

\begin{theorem}(Нормальность метризуемых пространств)
	Все метризуемые топологические пространства нормальны.
\end{theorem}
\begin{proof}
	Пусть $A$, $B$ --- непересекающиеся замкнутые множества. Тогда
	\begin{align*}
		\forall x \in A~ \exists r_x > 0\colon~ B(x, r_x) \cap B = \varnothing \\
		\forall y \in B~ \exists r_y > 0\colon~ B(y, r_y) \cap A = \varnothing
	\end{align*}
	Здесь мы воспользовались хаусдорфовостью метрических пространств.
	Положим
	\begin{align*}
		U = \bigcup_{x \in A}{B\left(x, \frac{r_x}{2}\right)} \\
		V = \bigcup_{y \in B}{B\left(y, \frac{r_y}{2}\right)}
	\end{align*}
	Эти множества открыты. Если мы докажем, что они не пересекаются,
	то множества $A$ и $B$ окажутся отделимыми.
	Пусть $U \cap V \neq \varnothing$, тогда (пусть $r_x \geqslant r_y$):
\begin{align*}
	&\exists r_x, r_y\colon~ B\left(x, \frac{r_x}{2}\right)
		\cap B\left(y, \frac{r_y}{2}\right) \neq 0 \Lra \\
	&\exists z\colon~ |xz| < \frac{r_x}{2},~ |yz| < \frac{r_y}{2} \Lra
	|xy| < \frac{r_x + r_y}{2} \leqslant \max(r_x, r_y) = r_x
	\Lra \\ &B(x, r_x) \ni y
\end{align*}
\end{proof}

\begin{remark}
	Т1, Т2, Т3 наследуются подпространством. Т4, вообще говоря, нет.
\end{remark}
